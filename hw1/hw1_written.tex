% Search for all the places that say "PUT SOMETHING HERE".

\documentclass[11pt]{article}
\usepackage{amsmath,textcomp,amssymb,geometry,graphicx,enumerate}

\def\Name{Shenao Zhang}  % Your name
\def\SID{3034487184}  % Your student ID number
\def\Homework{1} % Number of Homework
\def\Session{Spring 2019}


\title{CS188--Spring 2019 --- Homework \Homework }
\author{\Name, SID \SID}
\markboth{CS188--\Session\  Homework \Homework\ \Name}{CS188--\Session\ Homework \Homework\ \Name}
\pagestyle{myheadings}
\date{}

\newenvironment{qparts}{\begin{enumerate}[{(}a{)}]}{\end{enumerate}}
\def\endproofmark{$\Box$}
\newenvironment{proof}{\par{\bf Proof}:}{\endproofmark\smallskip}

\textheight=12in
\textwidth=6.5in
\topmargin=-.75in
\oddsidemargin=0.25in
\evensidemargin=0.25in


\begin{document}
\maketitle

\noindent\textbf{Due:} Monday 2/4/2019 at 11:59pm (submit via Gradescope).\\
Leave self assessment boxes blank for this due date.\\
\textbf{Self assessment due:} Monday 2/11/2018 at 11:59pm (submit via Gradescope)
For the self assessment, \textbf{fill in the self assessment boxes in your original submission} (you can
download a PDF copy of your submission from Gradescope). For each subpart where your original answer
was correct, write “correct.” Otherwise, write and explain the correct answer.\\
\textbf{Policy:} Can be solved in groups (acknowledge collaborators) but must be written up individually.\\
\textbf{Submission:} Your submission should be a PDF that matches this template. Each page of the PDF should
align with the corresponding page of the template (page 1 has name/collaborators, question 1 begins on page
2, etc.). \textbf{Do not reorder, split, combine, or add extra pages. }The intention is that you print out the
template, write on the page in pen/pencil, and then scan or take pictures of the pages to make your submission.
You may also fill out this template digitally (e.g. using a tablet.)\\
\\
\begin{center}
\begin{tabular}{|c|c|}
\hline 
First name&Shenao\\
\hline 
Last name&Zhang\\
\hline 
SID&3034487184\\
\hline
Collaborators&None\\
\hline
\end{tabular}
\end{center}
\newpage
\section*{Q1.Search}
\begin{qparts}
\item
\\
\begin{center}
\begin{tabular}{|c|c|c|c|}
\hline 
Search Algorithm&A-B-D-G&A-C-D-G&A-B-C-D-F-G\\
\hline 
Depth first search&\ &\ &$\surd$\\
\hline 
Breadth first search&$\surd$&\ &\ \\
\hline
Uniform cost search&\ &\ &$\surd$\\
\hline
A* search with heuristic $h_1$&\ &\ &$\surd$\\
\hline
A* search with heuristic $h_2$&\ &\ &$\surd$\\
\hline
\end{tabular}
\end{center}
\item
\begin{enumerate}[(i)] 
\item
 A heuristic h is admissible (optimistic) if:0$\leq$h(n)$\leq$x(n),where x(n) is the true cost to a nearest goal. So  $h_3(B) \in [0,13]$, where 13 denotes 5+3+5.
\item
Because h(A)-h(B)$\leq$cost(A to B)=1 and h(B)-h(c)$\leq$cost(B to C)=1, h(A)=10,h(C)=9, then $h_3(B) \in [9,10]$.
\item
$h_{3}(A)+g(A)=10+0=10,h_{3}(C)+g(C)=9+4=13,h_{3}(B)+g(B)=H_{3}(B)+5,h_{3}(D)+g(D)=7+10=17$, and if it is in order, then $13\leqH_{3}(B)+5\leq17$, then $h_3(B) \in [8,12]$.
\end{enumerate}
\newpage
\end{qparts}
\section*{Q2.n-pacmen search}
\begin{qparts}
\item
M tuples $((x_1,y_1),(x_2,y_2),...(x_M,y_M))$encoding the x and y coordinates of each pacman.
\item
The number of pacmen:n\\
Number of squares where pacmen can go:M\\
So  the state space is $M^n$.
\item
$M^n$.
\item
Because there is a closed set, the same node can not be expanded twice. The bound is $M^n$.
\item
\begin{enumerate}[(i)]
\item
Not consistent and not admissible.\\
Consider the situation that a square that there are three pacmen on its left, right, down. And the actual cost from this situation to goal is 1.(All of them move to the middle square) But  $h_1=1+1+1=3>1$. So $h_1$ is not consistent and not admissible.
\item
Consistent and admissible.\\
Because the every pacman moves by at most one unit vertically or horizontally at each time step. So $1/2*2\leq1$.And h(A)-h(B)$\leq$cost(A to B), satisfies consistent and admissible
\end{enumerate}
\end{qparts}
\end{document}
